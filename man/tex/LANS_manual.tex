\documentclass[a4paper, 11pt]{article}

\usepackage{caption}
\DeclareCaptionLabelFormat{adja-page}{\hrulefill\\#1 #2 {(see previous page)}}

\usepackage{graphicx}
\usepackage[export]{adjustbox}
\usepackage[left=1in,right=1in,top=1in,bottom=1in]{geometry}
\usepackage{xcolor}
\usepackage[bookmarksopen=true,pdfauthor=Lubos Polerecky,pdftitle=Look@NanoSIMS,pdfsubject=User manual]{hyperref}
\hypersetup{colorlinks=true,linkcolor=blue,urlcolor=blue}
\linespread{1.2}

\title{{\LARGE \bf Look@NanoSIMS}\footnote{\textbf{Citation}: L. Polerecky et al. (2012). Look@NanoSIMS --- a tool for the analysis of nanoSIMS data in environmental microbiology. \textit{Environmental Microbiology} 14 (4): 1009--1023. DOI: 10.1111/j.1462-2920.2011.02681.x}}
\author{\large\bf Lubos Polerecky\footnote{Feedback and questions can be sent to: l.polerecky@uu.nl}\\[3mm]
Max-Planck Institute for Marine Microbiology, Bremen, Germany (2002--2013)\\[2mm]
Department of Earth Sciences, Utrecht University, Utrecht, The Netherlands (2013--2025)}
\date{User's manual, version 28-06-2025\\[3mm]
\url{https://github.com/lpolerecky/LANS}}

\definecolor{darkgold}{RGB}{120,75,4}
\definecolor{darkgreen}{RGB}{4,120,4}
\definecolor{purple}{RGB}{128,0,128}
\newcommand{\ttt}[1]{\texttt{#1}}
\newcommand{\lans}[1]{{\color{magenta}#1}}
\newcommand{\lanscb}[1]{{\color{darkgreen}#1}}
\newcommand{\lanstf}[1]{{\color{cyan}#1}}
%\newcommand{<name>}[<args>]{ <code> }
\usepackage{marginnote}
\newcommand\mnote{\marginnote{\fbox{\textbf{\bf Note}}}}
\newcommand\ra{\rightarrow}
\newcommand\addon[1]{-- {\small #1}}
\newcommand\figref[0]{\textbf{Figure}}
%\newcommand\s[1]{\noindent\textbf{Step #1:}}
\newcounter{step}
\setcounter{step}{0}
%\newcommand\s{\addtocounter{step}{1}\paragraph{Step \thestep:}\hspace{-2mm}}
\newcommand\s{\addtocounter{step}{1}\noindent\textbf{Step \thestep:}{ }}
\newcommand\bul{\noindent$\bullet${ }}
\newcommand\bb[1]{\textbf{#1}}

\begin{document}

\maketitle
\reversemarginpar 

%%

\section*{Summary}
\textbf{Look@NanoSIMS}, or shortly \textbf{LANS}, is a program for the analysis and processing of nanoSIMS data. Its primary aim is to be \emph{useful}.  The original development of the program started back in 2008, so in many aspects the program may be considered as an `old school'. However, based on input and requests from users, the program has been improved in many ways, and an active development continues until today. Over the years, the program has matured well and can now be used for many types of analysis of nanoSIMS data. This document describes how such analyses can be done at basic, intermediate and very advanced levels.

%%

\tableofcontents

\section{Installation instructions}

LANS is a Matlab-based program. Thus, a core \textbf{Matlab} installation, along with image processing and statistical toolboxes, is required to run LANS and perform basic functions. This makes it possible to run LANS on a variety of operating systems, including Linux, Microsoft Windows and MacOS. However, this also limits LANS to users with access to a Matlab license. Additionaly functionality of LANS is achieved by integrating the program with \textbf{\LaTeX} (for exporting results in a nicely formatted PDF output) and data compression programs such as \textbf{zip} (for decompressing input files and compressing output generated by LANS), both of which are available for free.

%%

\subsection{Install Matlab}

\begin{enumerate}

\item Matlab is available from \url{www.mathworks.com} and requires a license. It is useful to check whether your institution has a site-license (e.g., your university may have one for all students and academic staff). 

\item Version 2019b of Matlab is most recommended to ensure that all features of LANS work as designed. LANS works with newer Matlab versions as well, however, there is a risk that some functionality will issue errors due to less than perfect back-compatibility of Matlab.

\item When installing Matlab, you will need the \emph{core Matlab} and \emph{two toolboxes}: image processing and statistics and machine learning. 

\end{enumerate}

\mnote
Some output generated by LANS, e.g., information generated during the alignment of planes and stored in the file \ttt{xyalign.mat}, may not be read correctly by the Matlab version 2019b when it was generated by a newer Matlab version. Thus, if you plan to use LANS in collaboration with other people, e.g., by sharing the files generated by LANS among each other, it is recommended that everyone in the team uses the same Matlab version.

%%

\subsection{Install \LaTeX}

This software is required to enable export of graphical output as tagged PDF documents. 

\begin{enumerate}
 
\item To install \LaTeX, use one of the well-established \LaTeX\ distributions for your operating system, as described on the \href{https://www.latex-project.org/get/}{\LaTeX\ project} website (e.g., \ttt{texlive} for Linux, \ttt{MikTeX} for Windows, \ttt{MacTex} for MacOS). Note that the on-line LaTeX service, such as Overleaf, is insufficient; you really need a locally installed \LaTeX\ distribution.
 
\item To correctly integrate \LaTeX\ with LANS, you will need the following executables and packages installed and working:
 
\begin{itemize}
\item executables: \ttt{epstopdf}, \ttt{pdflatex}
\item packages: \ttt{graphicx}, \ttt{geometry}, \ttt{hyperref}, \ttt{adjustbox}
\end{itemize}
 
\end{enumerate}
 
\mnote
If you have never used \LaTeX\ on your computer, it may be that some \LaTeX\ packages, particularly \ttt{geometry} and \ttt{hyperref}, are not installed during the 'standard' installation procedure. As a result, the execution of the LANS functions \lans{Export LaTeX and PDF output} (main LANS) or \lans{Export images for each variable as PDF} (Process metafile) may get stuck if the packages are missing. If this happens, you can fix this problem by compiling the \ttt{tex} file using the native \LaTeX\ environment (e.g., open the \ttt{tex} file in the default editor of your \LaTeX\ distribution and then compile it into a \ttt{pdf} output from there). When doing so, the missing \LaTeX\ packages should automatically be installed and the \ttt{tex} file should compile into a correct \ttt{pdf} output. Once this is done, the automatic \LaTeX\ compilation from within Matlab will also work.

%%

\subsection{Install software for compressing/decompressing files}

This software is required for two reasons.

\begin{enumerate}
 
\item It enables you to load compressed nanoSIMS datasets (\ttt{im.zip} files) by LANS. This is a useful feature because \ttt{im.zip} files have roughly a 10-fold smaller size than the original \ttt{im} files generated by the Cameca's nanoSIMS measurement software. It is recommended to store and distribute the raw data files by first compressing them with the \ttt{zip} program (extension \ttt{im.zip}). 

\item It allows you to compress the processed data generated by LANS. This is convenient for making data backups, since it is much more efficient to upload and download few compressed folders than hundreds of smaller files present in those folders.

\item \ttt{7-Zip} (freeware) is recommended for Microsoft Windows. \ttt{zip} and \ttt{unzip} are available by default on Linux and MacOS systems.

\end{enumerate}

%%

\subsection{Install Look@NanoSIMS}

This is done by copying the source files to a folder on your computer.

\begin{enumerate}
 
\item For convenience, the compressed file containing the \emph{latest version of LANS} is stored in this \href{https://www.dropbox.com/sh/gyss2uvv5ggu2vl/AABViAmt9WHryEP_xZBrCG_La?dl=0}{Dropbox folder}. Click on the \ttt{program} folder and then download the file \ttt{LANS-latest-src.zip}.

\item Unzip \ttt{LANS-latest-src.zip} to a folder of your choice.

\item Rename the \ttt{src} folder using a more reasoname name (e.g., \ttt{LANS-2025-05-26}, where the data will refer to the LANS version).

\end{enumerate}

\mnote
In case the Dropbox link above does not work, because it became outdated or the official distribution location changed, visit the LANS' github repository or try to search the internet for more updated information. For users familiar with git and github, LANS can be downloaded by pulling the source code from the \ttt{src} folder in the LANS github repository: \url{https://github.com/lpolerecky/LANS}.

%%

\subsection{Starting Look@NanoSIMS for the first time}

Before you run LANS for the first time, revise the content of the files \ttt{lookatnanosims.m} and \ttt{lans\_paths.m}. These two files contain important settings you need to adjust to reflect your specific local installation of LANS. For example, you can specify there:

\begin{itemize}
\item locations of the compression/decompression software,
\item location of the PDF viewer,
\item default name of the file containing regions of interest (ROIs),
\item default extension of the raw data files.
\end{itemize}

\mnote
If you browse through the LANS installation files, you will notice that the \ttt{*.fig} files, which define the graphical user interface (GUI), appear in two sub-folders: \ttt{figs} and \ttt{figs\_win}. This is required because GUI defined on Unix-like and Windows platforms do not look the same. This is an issue due to --- apparently --- limited cross-platform compatibility of Matlab visual objects. It is not important for you as an end-user of LANS. You only need to be aware of it. Should you wish to modify any of the \ttt{*.fig} files, you will need to do it twice.

%%

\subsection{Starting Look@NanoSIMS on a regular basis}

\begin{enumerate}

\item Start Matlab.

\item Set the current folder to the folder where you installed LANS. You can do this through the menu or, more easily, entering one of the following commands in the Matlab console (the precise syntax depends on whether you use Windows, Linux or MacOS, and on the path where you installed LANS):

\ttt{>> cd c:/programs/LANS}

\ttt{>> cd /home/your\_username/programs/LANS}

\item Once you are in the correct folder, enter the following command in the Matlab console:

\ttt{>> lookatnanosims}

If everything is set up correctly, the main LANS graphical user interface (GUI) will open (Fig.~\ref{fig1:mainLANSgui}). You can start from there, as described in the following sections of this document. 

\end{enumerate}

\begin{figure}[!t]
\centering
\includegraphics[width=0.9\textwidth]{figs1/LANS-maingui}
\caption{\label{fig1:mainLANSgui}%
Main graphical user interface (GUI) of Look@NanoSIMS.}
\end{figure}

\mnote
During the data processing session, LANS provides quite a lot of useful information in the Matlab console. Thus, it is a good idea to \emph{always} keep an eye on the output in the console. You can do this by arranging your desktop such that the LANS and Matlab console windows are \emph{both visible at the same time}. This is also useful in case you encounter an error while working in LANS. These errors will be shown in the console, too.


%%

\subsection{Updating Look@NanoSIMS}

\begin{enumerate}
 
\item LANS is updated quite regularly. You can update it easily by entering in the Matlab console:

\ttt{>> lans\_webupdate} 

You need to be in the folder where LANS is installed. You will be prompted to make a backup of your older LANS version, which is recommended to do, just in case.

\item If you are familiar with \ttt{git}, you can update LANS by pulling the latest sources from the LANS github repository \url{https://github.com/lpolerecky/LANS}.

\end{enumerate}

%%%%

\section{Organization of the input and output data}
\label{sec:data_organization}

Working with LANS, and with nanoSIMS data in general, can be a book-keeping challenge. To start with, you will have many raw data files (\ttt{im} or \ttt{im.zip} files) acquired at different dates and from different types of samples (e.g., different treatments). Additionally, processing of those raw data files with LANS will create many output files, including ASCII data, PDF images, matlab output, PDF output, and zipped backed-up folders. It is therefore a good idea to develop and maintain a certain structure of those folders and output files, to \emph{keep everything organized}. 

In this document, we assume that the raw and processed nanoSIMS data are organized hierarchically as shown in Table~\ref{tab1:file_structure}. We have used this data organization at Utrecht University for many years, and it works pretty well. We therefore highly encourage users to adopt it as well. Its benefits will become more apparent later on, when we get to the point of explaining how to efficiently process and analyze \emph{multiple} nanoSIMS datasets from a particular project.

\begin{table}[ht]
\centering
\caption{\label{tab1:file_structure} Hierarchical organization of the raw and processed nanoSIMS data implemented in Look@NanoSIMS. An example of such data organization along with a more detailed explanation is shown in Fig.~\ref{fig2:data_organization}.}
\begin{tabular}{l@{ $\rightarrow$ }c@{ $\rightarrow$ }l@{ $\rightarrow$ }l@{ $\rightarrow$ }l}
\hline
root & project & measurment day 1 & raw dataset 1.1 & \color{red}{dataset folder} 1.1\\
\multicolumn{2}{c}{} & & raw dataset 1.2  & dataset folder 1.2\\
\multicolumn{2}{c}{} & & $\cdots$ & $\cdots$ \\
\multicolumn{1}{c}{} & & measurement day 2 & raw dataset 2.1  & dataset folder 2.1\\
\multicolumn{2}{c}{} & & raw dataset 2.2  & dataset folder 2.2\\
\multicolumn{2}{c}{} & & $\cdots$ & $\cdots$\\
\multicolumn{1}{c}{} & & $\cdots$ & $\cdots$ & $\cdots$\\
\hline
\multicolumn{2}{r}{\color{red}{dataset folder} $i$ $\rightarrow$} & \color{orange}{dat} & \multicolumn{2}{l}{\color{orange}{numbers}} \\
\multicolumn{2}{r}{$\rightarrow$} & \textcolor{darkgold}{pdf} & \multicolumn{2}{l}{\textcolor{darkgold}{images \&\ graphs}} \\
\multicolumn{2}{r}{$\rightarrow$} & \multicolumn{3}{l}{\hspace{-2mm}processing definition files (\textcolor{purple}{alignment, ROIs, preferences})} \\
\multicolumn{2}{r}{$\rightarrow$} & \multicolumn{3}{l}{\hspace{-2mm}\textcolor{purple}{OutputG.pdf (output summary)}}\\
\hline
\end{tabular}
\end{table}

At the highest level of the data organization is a root folder that contains \emph{all} nanoSIMS data. This folder contains `project folders' with nanoSIMS data belonging to \emph{specific projects}. Each `project folder' contains `day folders' with data acquired on \emph{different measurement days}. Each `day folder' contains the actual \emph{raw datasets} (\ttt{im} or \ttt{im.zip} files). When a particular raw dataset is processed and analyzed, the corresponding data is stored in a `dataset folder' with the \emph{same name} as the dataset. Each `dataset folder' contains sub-folders with the \emph{results} of the analysis, including numerical values such as ROI-specific ion counts or ion count ratios (folder \ttt{dat}), and graphical output such as images or scatter plots (folder \ttt{pdf}). The `dataset folder' also contains information defining the processing steps, such as alignment of frames, definition and classification of regions of interest (ROIs), and preferences. This information is useful if you want to go back to the analysis of the same dataset after you have analyzed a different one, e.g., to perform quality checks or more in-depth analyses. Finally, the `dataset folder' also contains a PDF file with a comprehensive graphical summary of the analysis. This file is useful if you wish to share the results for a specific dataset with project collaborators.


\begin{figure}[b!]
\centering
\includegraphics[height=0.97\textheight]{figs2/folders_organization}
\caption[]{(see next page)}
\end{figure}
\begin{figure} [t!]
    \captionsetup{labelformat=adja-page}
    \ContinuedFloat
    \caption[Figure]{%
    Example of a hierarchical organization of the raw and processed nanoSIMS data implemented in Look@NanoSIMS. %
    \textbf{(A)} The root folder (\ttt{nanosims}) contains a~project-specific sub-folder (\ttt{GAP2017}), which contains sub-folders with data measured on different days (e.g., \ttt{2017-09-28-GAP2017}, \ttt{2017-10-09-GAP2017}). %
    The `day folder' contains \emph{multiple raw datasets} measured on that particular day (\ttt{im} or \ttt{im.zip} files). %
    When a particular raw dataset is processed and analyzed, the data is stored in a `dataset folder' with the \emph{same name} as the dataset (see red `!'). %
    \textbf{(B)} The `dataset folder' contains files defining the processing steps, including alignment information (\ttt{xyalign}), definition and classification of regions of interest (\ttt{cells.mat} and \ttt{cells.dat}, respectively), preferences (\ttt{prefs.mat}), and a comprehensive summary of results exported in a PDF file (\ttt{OutputG.pdf}). %
    The `dataset folder' also contains sub-folders with ASCII (\ttt{dat}), PDF (\ttt{pdf}) and Matlab (\ttt{mat}) output. %
    \textbf{(C)} The \ttt{dat} folder contains the ROI-specific ion counts and ion count ratios (\ttt{dac} files). %
    \textbf{(D)} The \ttt{pdf} folder contains the exported images, graphs, histograms, etc. }
    \label{fig2:data_organization}
\end{figure}

%%

\clearpage

\section{A typical data processing session with Look@NanoSIMS}

This section provides a \emph{quick summary} of steps taken during a typical data processing session with Look@NanoSIMS. Refer to Sections~\ref{sec:level1} and \ref{sec:level2} for more details, depending on the level of analysis. Throughout this document, we will use colored text when refering to a specific \lans{menu item or action button}, \lanscb{checkbox} or \lanstf{text field} in the program's GUI.


\subsection{Analysis of a single dataset --- ``from scratch''}
\label{sec:analysis_from_scratch}

The following steps will bring you from a raw dataset to basic output such as images or ROI-specific values of ion counts and ion count ratios. More details are provided in Section~\ref{sec:level1}.

\begin{enumerate}

\item Select \lans{Input} $\ra$ \lans{dead-time and QSA correction settings} to enable dead-time and QSA (quasi-simultaneous arrival) corrections.

\item Select \lans{Input} $\ra$ \lans{Load RAW dataset} to load raw data from disk. The raw data is stored in an \ttt{im.zip} or \ttt{im} file depending on whether it has been compressed or not.

\item Select \lans{Input} $\ra$ \lans{Autoscale plane images} to automatically set the scale for all masses. 

\item Select \lans{Input} $\ra$ \lans{Display plane images for all masses} to view the raw data as images, one plane at a time, displayed with a scale specified in the previous step. 

\item Define \lanstf{base mass for alignment} and then select \lans{Input} $\ra$ \lans{Display alignment mass} to check the data plane by plane, \lanscb{deselect planes} with artifacts or corrupt data, and \lans{define a region} for performing a drift-correction. 

\item Select \lans{Input} $\ra$ \lans{Accumulate plane images} to accumulate drift-corrected ion counts over selected planes.  

\item Select \lans{Input} $\ra$ \lans{Autoscale accumulated images} and then \lans{Input} $\ra$ \lans{Display accumulated images for all masses} to display the accumulated ion count images for all masses and export them in a PNG file.

\item Define \lanstf{expressions} for ion count ratios and the corresponding \lanstf{scales}. 

\item If you want to quantify ion counts and ion count ratios in regions of interest (ROIs), select \lans{ROIs} $\ra$ \lans{INTERACTIVE ROIs definition tool} to define the ROIs and store them on disk. Additionally, you can clasify the ROIs via \lans{ROIs} $\ra$ \lans{Classify} $\ra$ \lans{manually} or \lans{automatically}.

\item Select \lans{Output} $\ra$ \lans{Display masses} or \lans{Output} $\ra$ \lans{Display ratios} to perform various types of data analysis and visualization, including:

\begin{itemize}
\item display \lanscb{images}, \lanscb{depth profiles in ROIs}, \lanscb{lateral profiles}, or \lanscb{histograms},
\item \lanscb{combine images as RGB} overlays,
\item \lanscb{plot x-y-z graphs} (scatter plots) of ROI-specific ion counts or ratios,  
\item \lanscb{compare ROIs} with respect to ion counts or ratios using simple statistical methods.
\end{itemize}
%
Note that the appearance of the output can be tweaked via \lans{Preferences} $\ra$ \lans{Additional output options}.

%\addon{The results displayed will depend on the options selected in the \ttt{Output options} area of LANS. For example, you can display \lanscb{images}, \lanscb{depth profiles in ROIs}, \lanscb{lateral profiles}, or \lanscb{histograms}. You can also \lanscb{combine images as RGB} overlays, \lanscb{plot x-y-z graphs} (scatter plots), or \lanscb{compare ROIs} using simple statistical methods.}

%\addon{Displaying depth profiles of ion counts and ion count ratios in ROIs can be useful as a quality check of the data.}

%\addon{When displaying images, you can \lanscb{log-transform} the values, \lanscb{include ROI outlines}, or \lanscb{modulate the hue} based on a specific combination of masses (part of advanced analysis).}

%\addon{When exploring RGB overlays, including images and scatter plots, you can do so on a \lanscb{pixel-by-pixel} level or as \lanscb{ROI-averaged}.}

%\addon{Ensure that the checkboxes \lanscb{Export ASCII data} and \lanscb{Export PDF graphics} are checked if you want the results to be exported as numbers and images.}

\item Select \lans{Output} $\ra$ \lans{Generate LaTeX + PDF output} to export results of your analysis in a~comprehensive PDF file. 

%\mnote\addon{If you check \lanscb{View PDF after export} (available via \lans{Preferences} $\ra$ \lans{Additional output options}) and correctly set the \ttt{PDF\_VIEWER} variable (in the \ttt{lans\_paths.m} file), the PDF output will automatically be displayed after it has been generated. This can be convenient if you want to quickly see the results summary in a nicely organized way.}

\item Select \lans{Preferences} $\ra$ \lans{Store preferences} to save the settings of the current data processing session in a file (we recommend to use the file name \ttt{prefs.mat}). This is useful when you want to return to the analysis of the same file in the future, but also if you want to later combine results of your analyses of multiple datasets (``metafile processing''), as explained below. Therefore, it is highly recommended that you do this.

%\addon{This should be the last step of processing of an individual dataset. We emphasize that you do this step because the settings file will be \emph{very useful} if you want to re-process the current dataset in the future.}

%\addon{Specifically, the file contains all information for the currently processed dataset that you see in the GUI, including the detected masses, planes selected for alignment, formulas for ion count ratios, scales, all output options, etc.}

\item Optionally, select \lans{Preferences} $\ra$ \lans{Create full backup of processed data} to back up the processed data in a compressed (\ttt{zip}) file. This is useful for sharing the results of your data processing and analysis with collaborators.

\end{enumerate}

%%

\subsection{Continuation of a previously stored single-dataset analysis}

If you want to continue with the analysis of a dataset that you have analysed previously, you can save some time by following these steps:

\begin{enumerate}

\item Select \lans{Input} $\ra$ \lans{dead-time and QSA correction settings} to enable dead-time and QSA (quasi-simultaneous arrival) corrections.

\item Select \lans{Input} $\ra$ \lans{Load+accumulate+display RAW or PROCESSED dataset}. After you select the raw data and preferences files (\ttt{im.zip} or \ttt{im} file and \ttt{prefs.mat}, respectively), the data will be loaded, drift-corrected, accumulated and displayed automatically.

%\addon{You will be prompted to select the raw data (\ttt{im.zip} or \ttt{im} file) as well as the preferences file (e.g., \ttt{prefs.mat}). }

%\addon{After you do that, steps 2 (loading), 6 (drift-correction and accumulation) and 7 (display) described in Section \ref{sec:analysis_from_scratch} will be performed automatically.}
 
\item Select \lans{ROIs} $\ra$ \lans{Load ROIs from disk} to load the previously defined ROIs from disk.

%\addon{Choose the previously selected ROIs file, e.g., \ttt{cells.mat}. Modify ROIs definition or classification, if required.}

%\item Continue with steps 13--15 described in Section~\ref{sec:analysis_from_scratch}.

\end{enumerate}
%
After completing these steps, you can continue with further processing and analysis as described in the previous section (steps 8--13).

%%

\subsection{Analysis of multiple datasets --- ``metafile processing''}

After the analysis of several single datasets, you are left with isotope ratio values and images scattered across \emph{multiple} files and folders. You can quickly merge these multiple files into \emph{one} output file using the following steps. More details are provided in Section~\ref{sec:level2}.

\begin{enumerate}

\item Ensure that the raw and processed data are organized as described in Section~\ref{sec:data_organization} (see also Fig.~\ref{fig2:data_organization}).

\item Select \lans{Process multiple} $\ra$ \lans{Generate metafile} to define a list of datasets that will be analyzed together.

%\addon{Typically, the datasets selected in the list are part of a~project or correspond to a~particular set of treatments within a~project.}

\item Select \lans{Process multiple} $\ra$ \lans{Process metafile} to perform the analysis of multiple datasets.

\end{enumerate}
%
The output generated by metafile processing can be used for visualization and further analysis (e.g., statistical analysis) by LANS or other, third-party software.

%%

\section{Overview of advanced-level analyses in Look@NanoSIMS}

Over the years, different ways of analysis and visualization of nanoSIMS data have been added to Look@NanoSIMS. In this document, they are lumped into the category ``advanced level'', because they have been implemented to look at nanoSIMS data in a special or uncommon way. Many of them may not be very advanced at all, and most of them are likely useful only for a limited number of users. Still, it's good to know about them, just in case. Briefly, they include:

\begin{enumerate}

\item \lanscb{Hue modulation} of ratio images based on a combination of mass images.

\item Integrating an \lans{external image} into the nanoSIMS dataset. This includes \lans{alignment} and \lans{overlays} of external and nanoSIMS images, definition of \lanstf{ROIs based on the external image}, and \lans{resampling} of nanoSIMS data to match the resolution of the external image.

\item Loading datasets in \lans{blocks}, including the possibility to load and combine multiple raw datasets (measured in sequence) and analyze them as one dataset.

\item Visualization of \lanscb{lateral profiles across depth}.

\item Loading datasets with more than 8 masses. Such datasets can be acquired using the peak-switching mode, and they can contain up to 16 masses.

\item Simple statistical analysis of ROI-specific ratios, including correlation, ANOVA, Kruskal-Wallis test.

\end{enumerate}
%
More details about how to perform these analyses are provided in Section~\ref{sec:level3}.

%%%%

\section{Data processing with Look@NanoSIMS --- basic level}
\label{sec:level1}

This section describes in greater detail how to get from a raw dataset to basic output such as images or ROI-specific values of ion counts and ion count ratios. The dataset \ttt{2017-09-28-GAP2017\_1.im.zip} will be used as an example.

\addtolength{\parskip}{2mm}

\subsection{Load nanoSIMS dataset from disk}

\s Select \lans{Input} $\ra$ \lans{Dead-time and QSA correction settings} to enable dead-time and quasi-simul\-ta\-neous arrival (QSA) corrections. Define the relevant parameters in the window that pops up (\figref). The corrections will be \bb{applied} based on these parameters \bb{during loading} of the raw data. Note that you can specify the parameters for up to 16 masses. This can be relevant if the dataset was acquired using a peak-switching mode.

\s Select \lans{Ask for range of planes and masses before loading} if you want to interactively define \bb{a range of} planes and masses that should be extracted during loading of the raw dataset. This is unnecessary in most cases, but it may be useful if your dataset is huge (e.g., more than 1~GB of data) and the memory (RAM) on your computer is insufficient.

\s Select \lans{Shift columns or rows when loading raw data}. This data correction is relevant for data acquired by the NanoSIMS 50L instrument at Utrecht University and probably very few others around the world. It concerns a glitch in the acquisition software, which causes the data in pixels from the first column (or row) to appear in the last column (or row), or vice versa. This option allows you to \bb{shift} the pixels to the \bb{correct} position during loading of the raw dataset.

\s Select \lans{Load RAW dataset} to load the raw dataset from disk in a standard way (i.e., plane by plane). When searching for the dataset, choose the file type: \ttt{im.zip} or \ttt{im}, depending on whether the raw data has been compressed or not. After selecting the raw data file, observe the progress of loading in the Matlab console. When the loading is finished, names of the masses will automatically be filled in the corresponding \lanstf{mass} text fields. Additionally, \ttt{[]} will be added in the \lanstf{planes} text field, which stands for ``all planes''.

\addon{Sometimes, the mass name entry in the binary \ttt{im} file is incorrectly saved. If this happens, a~``strange-looking'' name such as \ttt{au197} will be filled as the corresponding \lanstf{mass}, referring to the actual mass detected (in atomic units). If you know that the detected isotope was (in this example) ${}^{197}$Au, you can rewrite the string from \ttt{au197} to \ttt{197Au} and use it as the name of the mass in further analysis.}

%%

\subsection{Display of mass images plane-by-plane}

\setcounter{step}{0}

\s Select \lans{Autoscale plane images} to automatically fill in the \lanstf{scale} for each detected mass. The scale, in the form of \ttt{[min max]}, is calculated as the 0.001 and 0.999 quantile of the ion counts across all pixels and planes. The quantiles can be adjusted using \lans{Preferences} $\ra$ \lans{Additional output options}. 

\s Select \lans{Display plane images for all masses} to view secondary ion counts detected in individual planes for all masses (\figref). This is the actual data in its \emph{rawest} form, i.e., ion counts per pixel per plane.

\bul Use arrows to skip through the planes forward and backward.

\bul Click on \lans{Export as PNG} to export all masses in a particular plane as a PNG file. If you additionally check \lanscb{Export all}, you can export all planes, each into a separate PNG file.

When exploring the ion count images, you can often notice a small \bb{drift} when going from one plane to the next. This drift is caused by temperature instabilities during the measurements (a~big jump can also be caused by an~earthquake!). You will be able to correct for this drift in the next step. At this point, you should decide on the mass that will be used to calculate the drift correction (called \lanstf{Base mass for alignment}). As a~rule of thumb, it should be one with a relatively \bb{high ion counts} and \bb{clear contrast} across the image. In this example, it will be \ttt{12C14N}.

%%

\subsection{Drift-correction and accumulation of planes}
\setcounter{step}{0}

The drift-corrected accumulation of planes is controlled by settings specified in the \ttt{Accumulation options} box and by the numbers in the \lanstf{planes} field (Figure). 

\s Specify the \lanstf{Base mass for alignment}, based on your choice in the previous step (in this example: \ttt{12C14N}). 

\s Select \lans{Input} $\ra$ \lans{Display alignment mass}. 

\bul In the window that pops up (Figure), click on \lans{arrows} to browse through the individual planes. 

\bul Check \lanscb{Deselect} to mark the plane that should be excluded from accumulation (e.g., if the data in that plane is corrupt).

\bul At the very end, select the \lans{alignment region} and close the window. The drift correction will be calculated based on this region. It is recommended to define it as a minimum rectangular region in the image that contains pronounced spatial heterogeneities (such as a cell or a group of cells).

\bul After you close the window, the x and y ranges of the alignment region will be written in the respective fields of \lanstf{Special region for alignment}. Also, the range of selected planes will be written in the \lanstf{planes} field. You can edit these fields if you know the correct values, but be careful not to make any syntax errors. 

\s Check \lanscb{Align images when accumulating} if you want to apply automated drift correction during the accumulation of planes.

\s Select \lans{Input} $\ra$ \lans{Accumulate plane images} to start the accumulation of drift-corrected images. 

\addon{You will be asked whether to employ a \lans{New} or \lans{Old} algorithm. In most cases, you will choose the new one. The old one is more robust if the drift-correction is based on images with very low ion counts (i.e., very pixelated images).}

\addon{Observe the drift-correction progress in the Matlab console. At the end of it, you will be prompted to accept or reject the drift-correction information.}

\addon{If you \lans{accept}, the information will be stored in the file \ttt{xyalign.mat} and all planes for all masses will be accumulated based on this information. This information will be used in the future if you load the raw dataset using \lans{Input} $\ra$ \lans{Load+accumulate+display RAW or PROCESSED dataset}.}

\addon{If you \lans{reject}, you can reiterate the previous steps (1--4) until you arrive at a dataset with accumulated planes.}

\addon{If you realize, anytime at a later point during your data analysis, that you are not satisfied with the drift-correction, you need to first delete the \ttt{xyalign.mat} file (via \lans{Preferences} $\ra$ \lans{Remove xyalign.mat from disk}) and then repeat the previous steps (1--4).}

%%

\subsection{Display accumulated mass images}
\setcounter{step}{0}

\s Select \lans{Input} $\ra$ \lans{Autoscale accumulated images} to update the \lanstf{scale} fields with a range optimized for displaying accumulated mass images. 

\addon{The optimum scale is calculated as the 0.001 and 0.999 quantiles of the accumulated ion counts across the image pixels. These quantiles can be adjusted via \lans{Preferences} $\ra$ \lans{Additional output options}.}

\s Alternatively, define the scale \bb{manually} by typing in the \ttt{[min max]} (minumum and maximum) values by yourself. When you then press \ttt{Enter}, the corresponding mass will be displayed in a new window using the color scale defined by the specified range. This is a \bb{quick way to adjust} the color scale for optimal display.

\s Select \lans{Input} $\ra$ \lans{Display accumulated images for all masses} to display the images, lumped together for all masses, and
export them in a PNG file.

\s Check the \lanscb{Display images} checkbox and then select \lans{Output} $\ra$ \lans{Display masses} to display the images in a nicer way and separately for each mass. 

\addon{The image appearance can be tweaked via \lans{Preferences} $\ra$ \lans{Additional output options}.}

\addon{In steps 3--4, you can choose whether you want to display the ion counts in a~linear or log scale by checking \lanscb{Log10-transform}.}

\addon{Check \lanscb{Export PDF graphics} to ensure that the displayed image is exported as PDF.}

%%

\subsection{Display ratio images}
\setcounter{step}{0}

\s Type \bb{formulas} for calculating ratio images in the  \lanstf{expression} fields. The formula can be any valid arithmetic expression containing any of the detected masses, such as:

\begin{itemize}
\item \ttt{13C/12C} to calculate the $^{13}C/^{12}C$ isotope ratio,
\item \ttt{13C/(12C+13C)} to calculate the $^{13}C$ atom fraction,
\item \ttt{32S/12C} to calculate the $^{32}S/^{12}C$ ion count ratio (a proxy for the S/C elemental ratio), 
\item \ttt{19F/plane} to calculate the average $^{19}F$ ion counts per detected plane, 
\item \ttt{19F/plane/pixel} to calculate the average $^{19}F$ ion count per plane per pixel (only applicable if ROIs are defined; see below).
\end{itemize}

\s For each expression, type in the \lanstf{scale} to the correspoding field. The scale should be in the form \ttt{[min max]} (minumum and maximum).

\addon{When you press \ttt{Enter} after you type the scale, the corresponding ratio image will be displayed in a new window using the color scale defined by the specified range. This is a \bb{quick way to adjust} the color scale for optimal display.}

\s Check the \lanscb{Display images} checkbox and then select \lans{Output} $\ra$ \lans{Display ratios} to display the ratio images.

\addon{The image appearance can be tweaked via \lans{Preferences} $\ra$ \lans{Additional output options}.}

%%

\subsection{Display RGB overlays}
\setcounter{step}{0}

Mass and ratio images can be combined into RGB overlays.

\s To specify the red, green and blue channels of the RGB overlay, enter identification numbers (1--8, found to the left of the mass names or ratio expressions) in the corresponding \lanstf{R}, \lanstf{G} and \lanstf{B} fields.

\addon{If one or more of the fields is left empty, the corresponding color channel(s) will be black.}

\s Check \lanscb{Combine images as RGB}.

\s Check \lanscb{Pixel by pixel} to combine the images without modification.

\s Check \lanscb{ROI-averaged} to use ROI-averaged values when creating the overlays.

\s Select \lans{Output} $\ra$ \lans{Display masses} or \lans{Display ratios} to display the RGB overlay derived from mass or ratio images, respectively.

\addon{If you want to overlay mass image(s) with ratio image(s), enter
the corresponding name(s) of the mass(es) into the \lanstf{expression} fields, enter the corresponding identification number in one of the \lanstf{color} fields, and select \lans{Output} $\ra$ \lans{Display ratios}.}

\addon{In addition to PDF, the RGB images are also exported as TIF files (in the \ttt{tif} sub-folder).}

%%

\subsection{Define regions of interest (ROIs)}
\setcounter{step}{0}

Typically, much of nanoSIMS data analysis revolves around ROIs. Look@NanoSIMS offers many options for defining ROIs, as described below. Once you get used to it and learn the ``tricks'', ROI definition can be very efficient.

\s Specify the \lanstf{ROI definition template}. 

\addon{This can be an individual mass, ratio, an RGB overlay of masses or ratios, or an external image.}

\addon{In this example, we will use an RGB overlay of \ttt{12C15N/12C14N}, \ttt{12C14N} and \ttt{31P}. Since we want to combine ratios and masses, we enter \ttt{12C14N} as expression \#3, \ttt{31P} as expression \#4, and copy the scale for both in the corresponding \lanstf{scale} field. Then, we enter \ttt{2}, \ttt{3} and \ttt{4} to the fields for \lanstf{R}, \lanstf{G} and \lanstf{B}, respectively. Finally, we enter \ttt{rgb2} as the \lanstf{ROI definition template} (entering \ttt{rgb} means that we use an RGB overlay, \ttt{2} means that we use an overlay of ratios).}

\addon{The ROI template can optionally be \bb{smoothed} before it is used for ROI definition. Smoothing is done using a median filter with a~two-dimensional kernel size (in pixels) specified in the \lanstf{Smoothing kernel} field. If applied, ROI outlines will generally be smoother if defined using the \lans{interactive thresholding} approach (see below). In this example, we will not do any smoothing and just use \ttt{[1 1]} as the smoothing kernel.}

\s Select \lans{ROIs} $\ra$ \lans{Display template for ROI definition} to verify that the template looks as intended. 

\s Select \lans{ROIs} $\ra$ \lans{INTERACTIVE ROIs definition tool} to open a~window dedicated to ROI definition (Figure). 

\addon{During this process, you will be prompted to select the \ttt{ROI file} (e.g., \ttt{cells.mat}) and \ttt{ROI classification file} (e.g., \ttt{cells.dat}). Select them if these files have been previously created and you want to reuse them, or press \lans{Cancel} if they do not exist (yet).} 

\addon{If they exist and you do select them both, they will be \bb{linked}. This means that when you add or remove a ROI from the image file, the corresponding ROI will also be added or removed from the classification file.}

The following steps refer to the \ttt{ROIs definition window}. When defining ROIs, you should \bb{always check messages} in the Matlab console. They will provide specific instructions or other relevant information. 

\setcounter{step}{0}

\s Select \lans{Action} $\ra$ \lans{Draw ROIs freehand} (\ttt{Ctrl+D}) to draw a ROI using a mouse.

\addon{Immediately after drawing a~ROI, you can still move it around using a mouse. \bb{Double-click} on the ROI to confirm its final position.}

\addon{When prompted, specify whether the ROI outline should be defined as the \lans{coarse polygon} you have just drawn or as an \lans{ellipse that circumscribes} this polygon.}

\s Experiment with different ways of defining ROIs, including drawing them as \lans{ellipses} (\ttt{Ctrl+E}) or \lans{rectangles} (\ttt{Ctrl+t}).

\addon{As before, you can use the mouse to resize the ROI and then double-click on the ROI to confirm it.}

\addon{Beware that if you draw a ROI over another, previously defined ROI, the last one will take the priority.}

\s Select \lans{Action} $\ra$ \lans{Interactive thresholding} (\ttt{Ctrl+A}) to define a ROI more efficiently. 

\addon{This approach is applicable if the signal used for recognizing the ROI stands out relative to the surrounding background.}

\addon{When an RGB overlay is used as a template for ROI definition, the color channel used for automatic ROI drawing can be changed via \lans{Select interactive channel} $\ra$ \lans{red}, \lans{green} or \lans{blue}.}

\addon{After invoking \lans{Interactive thresholding}, click with a left-mouse button on the image in a pixel with a high signal. You will notice that a ROI outline is drawn. The outline corresponds to a contour where the signal values do not fall below the value in the selected (clicked) pixel multiplied by a threshold value.}

\addon{The threshold value is 0.5 by default, but it can be changed interactively by pressing the \lans{up} or \lans{down} arrow key, respectively. When doing so, the ROI outline will cover a larger or smaller area. If a different pixel is selected, the ROI outline is automatically redrawn according to the current threshold. By combining mouse clicks with \lans{up}/\lans{down} arrows, you can rapidly optimize the ROI definition.}

\addon{When you are satisfied with the defined ROI, press \ttt{Enter} to confirm it. Press \ttt{Esc} if you want to cancel the current interactive ROI definition.} 

\addon{It is important to be aware that the \lans{interactive ROI definition} sequence of actions must \bb{always} be terminated properly by pressing either \ttt{Enter} or \ttt{Esc}. If this is not done, the program will proceed in an unexpected or confusing way and may need to be terminated ``the hard way''. Thus, \bb{always} watch additional instructions in the Matlab console.}

\s Select \lans{Display ROIs} $\ra$ \lans{Display ROIs with ROI ID's} (\ttt{Ctrl+W}) to display the currently defined ROIs.

\addon{If the ROIs have been classified, you can display only ROIs of a~specific class by typing the corresponding class identification letter in the dialog that opens.}

\addon{Notice that ROI identification numbers (ID's) are generated automatically and increase from left to right when sorting the ROIs based on their central point.}

\addon{If you want to display ROIs with specific ID's, you can do so by typing the ID's in the dialog that opens.}

\s Merging of multiple ROIs can be done via \lans{Action} $\ra$ \lans{Merge multiple ROIs into one}. 

\addon{To do this, you will need to know the updated ROI ID's. They can be obtained by displaying the ROIs, as explained in the previous step.}

\s Explore the \lans{Action} menu to test other ways to \lans{define}, \lans{split} or \lans{remove} ROIs (Figure).





%\item If relevant, select \lans{ROIs} $\ra$ \lans{Classify} $\ra$ \lans{manually} or \lans{automatically} to classify the defined ROIs. 

%\addon{This can be done \lans{manually} or \lans{automatically} based on ROI-specific information.}



%%

\section{Data processing with Look@NanoSIMS --- intermediate level}
\label{sec:level2}

%%

\section{Data processing with Look@NanoSIMS --- advanced level}
\label{sec:level3}

% earlier text:

%\addon{If an external image is chosen as a ROI definition template, it needs to be first aligned with the nanoSIMS image, which can be done via the \lans{External} menu (part of advanced analysis).}



\end{document}
