\section{Less common data processing in LANS --- an overview}

Over the years, different ways of analysis and visualization of nanoSIMS data have been added to Look@NanoSIMS. Although they may be useful only under special circumstances, it is good to know about them, just in case. For example, they include:

\begin{enumerate}

\item \lanscb{Hue intensity modulation} of ratio images based on rescaled ion counts.

\item Integrating an \lans{external image} into the nanoSIMS dataset. This includes \lans{alignment} and \lans{overlays} of external and nanoSIMS images, definition of \lanstf{ROIs based on the external image}, and \lans{resampling} of nanoSIMS data to match the resolution of the external image.

\item Loading datasets in \lans{blocks}, including the possibility to load and combine \emph{multiple} raw datasets (measured in sequence) and analyze them as one dataset.

\item Visualization of \lanscb{lateral profiles across depth}.

\item Loading datasets with more than 8 masses. Such datasets can be acquired using the peak-switching mode, and they can contain up to 16 masses.

\item Automatic ROI classification.

\end{enumerate}
%
More details about these analyses are provided in Section~\ref{sec:level3}.
